\documentclass{article}
\usepackage{amssymb, amsfonts, amsmath, amsthm}
\usepackage{enumerate}
\usepackage[colorlinks=true,
citecolor=black,
linkcolor=black,
anchorcolor=black,
filecolor=black,
menucolor=black,
urlcolor=black
]{hyperref}

\usepackage{epsfig,lpic,wrapfig}
\usepackage{multicol}

\usepackage[T1,T2A]{fontenc}
\usepackage[utf8]{inputenc}
\usepackage[
german,
english,
russian]{babel}

\usepackage[
sorting=none,
bibencoding=auto,
backend=biber,
maxnames=10,
autolang=other,
doi=false,
url=false,
style=numeric-comp,
isbn=false]{biblatex}
\usepackage{csquotes}
\AtEveryBibitem{\clearlist{language}}
\AtEveryBibitem{\clearlist{note}}
\renewbibmacro{in:}{}
\renewcommand*{\bibfont}{\scriptsize}
\addbibresource{library.bib}

\DeclareFieldFormat{postnote}{#1}
\DeclareFieldFormat{multipostnote}{#1}
\DeclareFieldFormat{pages}{#1}

\def\thetitle{Граф-сравниения и условия Александрова}
\def\theauthors{Н. Д. Лебедева и А. М. Петрунин}
\hypersetup{pdftitle={\thetitle},
pdfauthor={\theauthors}}

\def\parit#1{\medskip\noindent{\it #1}}
\def\parbf#1{\medskip\noindent{\bf #1}}
\def\qeds{\qed\par\medskip}
\def\qedsf{\vskip-6mm\qeds}

\renewcommand{\labelitemi}{$\mathsurround=0pt \diamond$}
\renewcommand{\labelitemii}{$\mathsurround=0pt \circ$}


\newcounter{thm}[section]
\renewcommand{\thethm}{\arabic{thm}}
\def\claim#1{\par\medskip\noindent\refstepcounter{thm}\hbox{\bf\boldmath #1.}
\it\ 
}
\def\endclaim{
\par\medskip}
\newenvironment{thm}{\claim}{\endclaim}

\newcommand*{\z}[1]{#1\nobreak\discretionary{}{\hbox{$\mathsurround=0pt #1$}}{}}


\begin{document}

\title{\thetitle}
\author{\theauthors}

\date{}
\maketitle
\begin{abstract}
Граф-сравнение это условие определённого типа на метрические пространства закодированное конечным графом.
Мы доказываем, что любое нетривиальное граф-сравнение влечёт одно из двух условий Александрова.
Доказательство также даёт классификацию графов с тривиальным сравнением.
\end{abstract}

\parbf{Предисловие.}
Граф-сравнения были введены в \cite{lebedeva-petrunin-zolotov} и изучалось далее в \cite{toyoda,toyoda2019,lebedeva-petrunin-CBB,lebedeva,lebedeva-petrunin,lebedeva-petrunin-octahedron}.
Перечислим некоторые из результатов.
\begin{itemize}
\item Граф-сравнения для треноги и четыре-цикла описывают условия Александрова (см. ниже).
\end{itemize}
\begin{itemize}
\item Граф-сравнения для звёзд дают более сильный вариант так называемых \emph{неравенств Лэнга --- Шрёдера --- Штурма} \cite{lang-schroeder, sturm, lebedeva-petrunin-CBB}.
\end{itemize}
\begin{itemize}
\item Граф-сравнения для всех деревьев даёт описание пространств-образов субметрий из подмножеств гильбертова пространства \cite{lebedeva-petrunin-zolotov}.
\end{itemize}

\noindent
\begin{minipage}
{.80\textwidth}
\begin{itemize}
\item Граф-сравнение для дерева на рисунке, тесно связанно с \emph{непрерывностью решения транспортной задачи} и с так называемым \emph{условием Ма --- Трудингера --- Ванга}~\cite{lebedeva-petrunin-zolotov,ma-trudinger-wang}.
\end{itemize}
\end{minipage}
\hfill
\begin{minipage}{.17\textwidth}
\centering
\vskip-1mm
\includegraphics{mppics/pic-5}
\end{minipage}
\begin{itemize}
\item Октоэдральное сравнение выполняется в произведении деревьев~\cite{lebedeva-petrunin-octahedron}.
\end{itemize}

Мы покажем, что любое нетривиальное граф-сравнение влечёт одно из двух условий Александрова.

\parbf{Введение.}
Начнём с определения.
Пусть $\Gamma$ --- граф с вершинами $v_1,\dots,v_n$.
Будем писать $v_i\sim v_j$ (или $v_i\nsim v_j$) если вершина $v_i$ соединена (соответственно не соединена) ребром с $v_j$.

Метрическое пространство $X$ удовлетворяет \emph{$\Gamma$-сравнению} если для любых $n$ точек в $X$ обозначенных $v_1,\dots,v_n$ существует конфигурация сравнения --- набор точек $\tilde v_1,\dots,\tilde v_n$ в гильбертовом пространстве $\mathbb{H}$ такой, что
\begin{align*}
v_i\sim v_j\quad&\Longrightarrow\quad|\tilde v_i-\tilde v_j|_{\mathbb{H}}\leqslant | v_i-v_j|_{X},
\\
v_i\nsim v_j\quad&\Longrightarrow\quad|\tilde v_i-\tilde v_j|_{\mathbb{H}}\geqslant | v_i-v_j|_{X},
\end{align*}
где $|\ -\ |_X$ обозначает расстояние в~$X$.
(Обратите внимание, что здесь $v_i$ относится и к вершине $\Gamma$ и к соответствующей точке $X$.)

\begin{wrapfigure}{r}{35 mm}
\vskip-0mm
\centering
\includegraphics{mppics/pic-10}
\end{wrapfigure}

Обозначим через $T_3$ и $C_4$ треногу и четыре-цикл показанные на рисунке.
Их сравнения эквивалентны соответственно неотрицательной и неположительной кривизне в смысле Александрова \cite{lebedeva-petrunin-zolotov}.
Эти определения обычно используются для пространств с внутренней метрикой,
однако их можно применять и к произвольным метрическим пространствам.
Последнее соглашение используется, например, в \cite{alexander2019alexandrov}.

\begin{thm}{Теорема}
Пусть $\Gamma$ --- конечный граф.
Тогда, либо $\Gamma$-сравнение выполняется в любом метрическом пространстве,
любо оно влечёт $C_4$- или $T_3$-сравнение.
\end{thm}

Утверждение ниже является следствием из доказательства теоремы;
оно описывает все графы с тривиальным сравнением.

\begin{thm}{Следствие}
Пусть $\Gamma$ --- конечный связный граф.
Предположим, что $\Gamma$-сравнение тривиально;
то есть, оно выполняется в любом метрическом пространстве.
Тогда граф $\Gamma$ можно построить из пути $P_{\ell}$ длины $\ell\geqslant 0$ и двух полных графов $K_{m_1}$, $K_{m_2}$ подсоединив $k_1$ вершин $K_{m_1}$ к левому концу $P_{\ell}$ и $k_2$ вершин $K_{m_2}$ к правому концу $P_{\ell}$.
\end{thm}

Граф $\Gamma$ в этом следствии определяется пятёркой целых чисел $(m_1$, $k_1$, $\ell$, $k_2$, $m_2)$ таких, что $\ell \geqslant 0$, $m_i\geqslant k_i\geqslant 0$, и $k_i>0$, если $m_i>0$ для всякого $i$. 
Примеры таких графов и их пятёрок показаны ниже.
\begin{figure}[ht!]
\centering
\includegraphics{mppics/pic-15}
\end{figure}

\bigskip

\parbf{Доказательство теоремы.}
Предположим, что граф $\Gamma$ распадается на связные компоненты $\Gamma_1,\dots,\Gamma_k$.
Заметим, что $\Gamma$-сравнение эквивалентно выполнению всех $\Gamma_i$-сравнений.
Таким образом можно предположить, что $\Gamma$ связен.

Положим, как и выше, что $v_1,\dots,v_n$ --- вершины $\Gamma$.
Удалим из $\Gamma$ пару вершин, скажем $v_1$ и $v_2$.
Добавим вместо них новую вершину $w$ так, что для любой другой вершины $u$ выполняется следующее
\begin{itemize}
 \item если $u\sim v_1$ и $u\sim v_2$, то $u\sim w$;
 \item если $u\nsim v_1$ и $u\nsim v_2$, то $u\nsim w$;
 \item в остальных случаях можно выбрать $u\sim w$ или $u\nsim w$.
\end{itemize}
Пусть $\Gamma'$ --- полученный граф.

Полагая в определении $\Gamma$-сравнения, что точки $v_1$ и $v_2$ совпадают в $X$,
мы приходим к следующему.

\begin{thm}{Утверждение}
$\Gamma$-сравнение влечёт $\Gamma'$-сравнение.

\end{thm}

Приведённое построение $\Gamma'$ из $\Gamma$ будет называться \emph{слиянием вершин}.
Если граф $\Delta$ можно получить из $\Gamma$ применив слияние вершин несколько раз, то мы будем писать $\Delta\prec \Gamma$.

Заметим, что вышесказанное влечёт следующие два наблюдения:
\begin{itemize}
 \item Если $\Delta$ --- индуцированный подграф конечного графа $\Gamma$ тогда $\Delta\prec \Gamma$.
 \item Если $\Delta\prec \Gamma$, то $\Gamma$-сравнение влечёт $\Delta$-сравнение.
\end{itemize}
Отсюда получаем следующую переформулировку теоремы.

\begin{thm}{Переформулировка}
Для любого конечного связного графа $\Gamma$, 
\begin{enumerate}[(a)]
\item $\Gamma$-сравнение тривиально или
\item $C_4\prec \Gamma$ или
\item $T_3\prec \Gamma$.
\end{enumerate}
\end{thm}

Связный граф будем называть \emph{мультипутём}, если существует целозначная функция $\ell$ определённая на его вершинах такая, что 
$$v\sim w
\quad\Longleftrightarrow\quad
|\ell(v)-\ell(w)|\leqslant 1.$$
Значение $\ell(w)$ будет называться \emph{уровнем} вершины $w$.
Мультипуть полностью определяется целочисленной последовательностью, задающей количество вершин на каждом уровне.
\begin{figure}[ht!]
\centering
\medskip
\includegraphics{mppics/pic-20}
\medskip
\end{figure}
Пример мультипути и его последовательность показаны на рисунке.

\begin{thm}{Лемма}
Если $\Gamma$ --- конечный связный граф такой, что $C_4\nprec\Gamma$ и $T_3\nprec\Gamma$,
то $\Gamma$ --- мультипуть.
\end{thm}

\parit{Доказательство.}
Обозначим через $|\ -\ |_\Gamma$ метрику кратчайшего пути на~$\Gamma$;
она выдаёт число рёбер в кратчайшем пути между двумя вершинами.
Докажем, что
\[|u-w|_\Gamma\geqslant |u-v|_\Gamma\geqslant|v-w|_\Gamma\geqslant 2
\quad\Longrightarrow\quad |u-w|_\Gamma=|u-v|_\Gamma+|v-w|_\Gamma
\leqno({*})\]
для любых трёх вершин $u$, $v$ и $w$ в $\Gamma$.

Предположим, что $({*})$ не выполняется для $u$, $v$ и $w$.
Рассмотрим подграф $\Delta$ в $\Gamma$, индуцированный тройкой кратчайших путей между $u$, $v$ и $w$.
\begin{figure}[ht!]
\centering
\includegraphics{mppics/pic-25}
\end{figure}
Заметим, что $\Delta$ либо является циклом,
либо он содержит три пути из некоторой вершины, скажем $o$, до  $u$, $v$ и $w$ таких, что каждый путь не заходит в оставшиеся вершины тройки.
В этих случаях имеем $C_4\prec\Delta$ или $T_3\prec\Delta$ соответственно.
Применив наблюдение выше, получаем $\Delta\prec\Gamma$ --- $({*})$ доказано.

Обозначим через $d$  диаметр $\Gamma$.
Можно предположить, что $d\geqslant 2$.
Если $d=1$, то $\Gamma$ --- полный граф, в частности, мультипуть.

Выберем вершины $p$ и $q$ такие, что $|p-q|_\Gamma=d$.
Покажем, что $\Gamma$ есть мультипуть со следующей функцией уровня
\[\ell(w)=
\begin{cases}
\hfil |p-w|_\Gamma&\text{если}\quad |p-w|_\Gamma\geqslant 2,
\\
d-|q-w|_\Gamma&\text{если}\quad |q-w|_\Gamma\geqslant 2,
\\
\hfil 1&\hfil\text{в остальных случаях.}
\end{cases}
\]
Из $({*})$ получаем, что $|p-w|_\Gamma+|q-w|_\Gamma=d$, если $|p-w|_\Gamma\geqslant 2$ и $|q-w|_\Gamma\geqslant 2$, а значит $\ell$ хорошо определена.

Если $d\geqslant 4$, то наше утверждение следует из $({*})$.
Остаётся два случая: $d=2$ и $d=3$.

\begin{wrapfigure}{r}{17 mm}
\vskip-6mm
\centering
\includegraphics{mppics/pic-26}
\vskip-2mm
\end{wrapfigure}

Заметим, что граф-веер на рисунке не может быть индуцированным подграфом $\Gamma$.
Действительно, применяя слияние вершин для концов отмеченного ребра можно получить треногу --- противоречие.

\parit{Случай $d=2$.}
По $({*})$, $\ell^{-1}(0)$ и $\ell^{-1}(2)$ образуют клики.
Заметим, что $\ell(v)=1$ тогда и только тогда, когда $p\sim v$ и $q\sim v$.
Далее заметим, что $\ell^{-1}(1)$ образует клику;
действительно, если $u\nsim v$ для некоторых $u,v\in\ell^{-1}(1)$,
то подграф индуцированный $\{p,q,u,v\}$ есть четыре-цикл --- противоречие.

Остаётся доказать, что $u\sim v$, $v\sim w$, и $u\nsim w$ если $\ell(u)=0$, $\ell(v)=1$, и $\ell(w)=2$.

\begin{wrapfigure}{r}{17 mm}
\vskip-2mm
\centering
\includegraphics{mppics/pic-27}
\vskip4mm
\includegraphics{mppics/pic-28}
\end{wrapfigure}

Предположим $u\sim w$.
Заметим, что $v\sim u$ и $v\sim w$, иначе $\Gamma$ содержит индуцированный цикл длины 4 или 5 с вершинами $p$, $u$, $v$, $w$, $q$.
Таким образом, подграф индуцированный $\{p,u,v,w,q\}$ изоморфен вееру --- противоречие.

Предположим $v\nsim w$.
Заметим, что $w\ne q$.
Обозначим через $m$ вершину между $w$ и $p$.
Из вышесказанного, $\ell(m)=1$;
в частности $v\sim m\sim q$.
И снова, подграф индуцированный $\{p,v,m,w,q\}$ изоморфен вееру  --- противоречие.
Точно так же доказывается, что $u\sim v$.

\parit{Случай $d=3$.}
Нам нужно доказать что $u\sim v$ если $\ell(u)=2$ и $\ell(v)=3$;
остальное следует из $({*})$.

\begin{wrapfigure}{r}{20 mm}
\vskip-0mm
\centering
\includegraphics{mppics/pic-29}
\vskip-2mm
\end{wrapfigure}

Предположим противное.
Пусть $u'$ ($v'$) лежит посередине $u$ и $q$ (соответственно, $v$ и $p$).
Заметим, что $\ell(u')=3$, $\ell(v')=2$, и $u'\sim v'$ (иначе подграф, индуцированный $\{u,v,u',v'\}$, образует 4-цикл).
То есть, подграф, индуцированный $\{p,q,u,v,u',v'\}$, изображён на рисунке.
Заметим, что он содержит веер --- противоречие.
\qeds


\begin{thm}{Предложение}
Пусть $\Gamma$ --- мультипуть с последовательностью $(k_0,\z\dots, k_m)$.
Предположим $C_4\nprec\Gamma$ и $T_3\nprec\Gamma$.
Тогда
\begin{enumerate}[(a)]
 \item\label{lem:multipath:5} Если $m\geqslant 4$, то $k_2=\dots=k_{m-2}=1$.
 \item\label{lem:multipath:4} Если $m= 3$, то $k_1=1$ или $k_2=1$.
 \item\label{lem:multipath:3} Если $m= 2$, то $k_0=1$, $k_1=1$ или $k_2=1$.
\end{enumerate}

\end{thm}

\parit{Доказательство.} Предполагая противное в каждом случае, получаем

\parit{(\ref{lem:multipath:5})}
если $m\geqslant 4$, то мултипуть $(1,1,2,1,1)$ --- индуцированный подграф в~$\Gamma$,

\parit{(\ref{lem:multipath:4})}
если $m=3$, то мультипуть $(1,2,2,1)$ --- индуцированный подграф в~$\Gamma$,

\parit{(\ref{lem:multipath:3})}
если $m=2$, то мультипуть $(2,2,2)$ --- индуцированный подграф в~$\Gamma$.

В каждом случае мы приходим к противоречию, применив слияние вершин к концам отмеченный рёбер на рисунке.

\begin{figure}[h!]
\centering
\includegraphics{mppics/pic-30}
\end{figure}
\qedsf

Остаётся показать, что для любого мультипути $\Gamma$, описанного в предложении,
$\Gamma$-сравнение тривиально.
Мы проделаем это явно, указав координаты конфигурации сравнения на вещественной прямой.

Каждое ребро в $\Gamma$ имеет вес --- расстояние между его концами в $X$.
Определим расстояние $\|v-w\|_\Gamma$ как минимальный вес пути, соединяющего $v$ и $w$ в $\Gamma$.
Заметим, что 
\[\|v-w\|_\Gamma\geqslant |v-w|_X\]
для любых $v$ и $w$.

Если $m\leqslant 1$, то $\Gamma$ --- полный граф.
В этом случае $\Gamma$-сравнение тривиально.
Остаётся рассмотреть случаи $m\geqslant2$.

Выберем вершину $w$, которая единственна на своём уровне и не очень далека от середины $\Gamma$.
А именно, если $m\geqslant 4$, то возьмём за $w$ вершину на втором уровне; 
согласно предложению, она в нём единственна.
Если $m=3$, то согласно предложению можно предположить, что $k_2=1$;
в этом случае выберем $w$ на втором уровне.
И наконец, если $m=2$, пусть $w$ будет единственной вершиной в своём уровне;
по предложению такая существует.

Для каждой вершины $v_i$, положим
\[\tilde v_i=\pm \|w-v_i\|_\Gamma,\]
здесь стоит знак плюс, если уровень $v_i$ выше уровня $w$, и минус в противном случае.
По неравенству треугольника, полученная конфигурация $\tilde v_1,\dots,\tilde v_n\in\mathbb{R}$ удовлетворяет $\Gamma$-сравнению.
\qeds

\parbf{Замечания.}
Утверждения, перечисленные в предисловии, говорят, что сравнение для правильно выбранного графа (или семейства графов) отвечает за содержательное семейство метрических пространств.
Далее мы приведём ещё пару наблюдений о граф-сравнении общего характера.

Граф-сравнения для всех полных двудольных графов влекут так называемые \emph{чистые неравенства отрицательного типа} \cite[6.1.1]{deza-laurent-ru}.
По критерию Шёнберга, этих неравенств достаточно для существования изометрического вложения в гильбертово пространство \cite[6.2.1]{deza-laurent-ru}.
В частности выполнение сравнений для всех графов влечёт, что метрическое пространство изометрично подмножеству гильбертова пространства.
Последнее утверждение можно доказать также и тем же путём, что \cite[Prop. 1.9]{toyoda}.


\begin{wrapfigure}{r}{17 mm}
\vskip-6mm
\centering
\includegraphics{mppics/pic-45}
\vskip-2mm
\end{wrapfigure}

Предыдущие наблюдения выполняются для произвольных метрик.
Для пространств с внутренней метрикой большинство граф-сравнений влекут то, что пространство изометрично подмножеству гильбертова пространства.
Действительно, если $C_4\prec\Gamma$ и $T_3\prec\Gamma$ (как для графа на рисунке),
то любое полное пространство с внутренней метрикой имеет нулевую кривизну в смысле Александрова; 
в частности изометрично выпуклому множеству гильбертова пространства.

\parbf{Благодарности.}
Мы благодарны А. Б. Лычаку за помощь. 

Первый автор частично поддержан Российским фондом фундаментальных исследований, грант 20-01-00070;
второй автор поддержан Национальным научным фондом, грант DMS-2005279
и
Министерством образования и науки Российской Федерации, грант 075-15-2022-289.

{\sloppy
\printbibliography[heading=bibintoc]
\fussy
}

{\bigskip\footnotesize

\noindent Н. Д. Лебедева,
\par\nopagebreak
 \textsc{Санкт-Петербургский государственный университет, Университетская наб., д. 7–9, Санкт-Петербург, 199034, Россия.}
\par
\nopagebreak
 \textsc{Федеральное  государственное  бюджетное  учреждение  науки
Санкт-Петербургское  отделение  Математического  института
им. В. А. Стеклова  Российской  академии  наук, наб. Фонтанки 27,
191023 Санкт-Петербург,
Россия }
  \par\nopagebreak
  \textit{Электронный адрес}: \texttt{lebed@pdmi.ras.ru}

\medskip

\noindent   А. М. Петрунин, 
\par\nopagebreak
 \textsc{Math. Dept. PSU, University Park, PA 16802, USA.}
  \par\nopagebreak
  \textit{Электронный адрес}: \texttt{petrunin@math.psu.edu}
  
}
\end{document}
