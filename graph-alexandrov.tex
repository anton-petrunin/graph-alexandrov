\documentclass{article}
\usepackage{quad}

\def\thetitle{Graph comparison meets Alexandrov}
\def\theauthors{Nina Lebedeva and Anton Petrunin}

\hypersetup{pdftitle={\thetitle},
pdfauthor={\theauthors}}

\newcommand{\Addresses}{{\bigskip\footnotesize

\noindent Nina Lebedeva,
\par\nopagebreak
 \textsc{Saint Petersburg State University, 7/9 Universitetskaya nab., St. Petersburg, 199034, Russia}
\par
\nopagebreak
 \textsc{St. Petersburg Department of V.A. Steklov Institute of Mathematics of the Russian Academy of Sciences, 27 Fontanka nab., St. Petersburg, 191023, Russia}
  \par\nopagebreak
  \textit{Email}: \texttt{lebed@pdmi.ras.ru}

\medskip

\noindent   Anton Petrunin, 
\par\nopagebreak
 \textsc{Math. Dept. PSU, University Park, PA 16802, USA.}
  \par\nopagebreak
  \textit{Email}: \texttt{petrunin@math.psu.edu}
  
}}

\begin{document}
%\pagestyle{empty}\renewcommand\includegraphics[2][{}]{}


\title{\thetitle}
\author{\theauthors}

\date{}
\maketitle
\begin{abstract}
Graph comparison is a certain type of condition on metric space encoded by a finite graph.
We show that any nontrivial graph comparison implies one of two Alexandrov's comparisons.
The proof gives a complete descriptions of graphs that have trivial graph comparison.
\end{abstract}

The notion of graph comparison introduced in \cite{lebedeva-petrunin-zolotov}.
It was studied further in \cite{toyoda,toyoda2019,lebedeva-petrunin-CBB,lebedeva,lebedeva-petrunin,lebedeva-petrunin-octahedron}.
Let us mention some of the results.
\begin{itemize}
\item Graph comparison captures nonnegative and nonpositive curvature in the sense of Alexandrov.
\item Graph comparison for certain trees is used to formulate a stronger version of the so-called \emph{Lang--Schroeder--Sturm inequality} \cite{lang-schroeder, sturm}.
\item The all-tree comparison it gives a metric description of target spaces of submersions from susets of Hilbert space.
\item For certain tree, graph comparison has tight relation with the so-called \emph{MTW condition} that was introduced by Xi-Nan Ma, Neil Trudinger and Xu-Jia Wang~\cite{ma-trudinger-wang}.
\item Octahedron comparison holds in products of trees.
\end{itemize}
We will show that any nontrivial graph comparison implies one of two Alexandrov's comparisons.

Let us start with the definition.
Suppose $\Gamma$ is a graph with vertices $v_1,\dots,v_n$.
We write $v_i\sim v_j$ (or $v_i\nsim v_j$) if $v_i$ is adjacent (respectively nonadjacent) to $v_j$.

A metric space $X$ meets the \emph{$\Gamma$-comparison} if for any $n$ points in $X$ labeled by vertices of $\Gamma$ there is a model configuration $\tilde v_1,\dots,\tilde v_n$ in the Hilbert space $\HH$ such that 
\begin{align*}
v_i\sim v_j\quad&\Longrightarrow\quad|\tilde v_i-\tilde v_j|_{\HH}\le | v_i-v_j|_{X},
\\
v_i\nsim v_j\quad&\Longrightarrow\quad|\tilde v_i-\tilde v_j|_{\HH}\ge | v_i-v_j|_{X};
\end{align*}
here $|\ -\ |_X$ denotes distance in the metric space~$X$.

\begin{wrapfigure}{r}{35 mm}
\vskip-6mm
\centering
\includegraphics{mppics/pic-10}
\end{wrapfigure}

Denote by $T_3$ and $C_4$ and tripod and four-cycle shown on the diagram.
The $C_4$-comparison is equivalent to nonnegative curvature,
and $T_3$-comparison are equivalent to the nonpositive curvature in the sense of Alexandrov \cite{lebedeva-petrunin-zolotov}.
These definitions usually applied to length spaces, but they can be applied to general metric spaces;
the latter convention is used in \cite{alexander2019alexandrov}.

\begin{thm}{Theorem}
Let $\Gamma$ be an arbitrary finite graph.
Then either $\Gamma$-comparison holds in any metric space,
or it implies $C_4$- or $T_3$-comparison.
\end{thm}

The proposition below gives a description of all graphs $\Gamma$ with trivial $\Gamma$-comparison.

\parit{Proof.}
Suppose $\Gamma$ has connected components $\Gamma_1,\dots,\Gamma_k$.
Observe that $\Gamma$-comparison holds in a metric space $X$ if and only if so does every $\Gamma_i$-comparison.
Therefore we can assume that $\Gamma$ is connected.

Suppose $\Gamma$ is a graph with vertices $v_1,\dots,v_n$ as before.
Let $e$ be an edge in $\Gamma$; we can assume that it connects $v_1$ to $v_2$.
Remove $v_1$ and $v_2$ from $\Gamma$ and add a new vertex $w$ such that for any other vertex $u$ we have
\begin{itemize}
 \item if $u\sim v_1$ and $u\sim v_2$, then $u\sim w$;
 \item if $u\nsim v_1$ and $u\nsim v_2$, then $u\nsim w$;
 \item in the remaining cases we can choose arbitrarily $u\sim w$ or $u\nsim w$.
\end{itemize}
Denote the obtained graph by $\Gamma'$.

Applying the definition of $\Gamma$-comparison with $v_1=v_2$, we get the following.

\begin{thm}{Claim}
If $\Gamma$-comparison holds in a metric space $X$, then so does $\Gamma'$-comparison.

\end{thm}

The operation that produce $\Gamma'$ from $\Gamma$ will be called \emph{edge shrinking}.
If a graph $\Delta$ can be obtained from $\Gamma$ applying edge shrinking several times, then we will write $\Delta\prec \Gamma$.

Note that the claim implies the following two statements:
\begin{itemize}
 \item If $\Delta$ is induced subgraph of a connected finite graph $\Gamma$, then $\Delta\prec \Gamma$.
 \item If $\Delta\prec \Gamma$, then $\Gamma$-comparison implies $\Delta$-comparison.
\end{itemize}
Taking all above into account, we get the following reformulation of the theorem.

\begin{thm}{Reformulation}
For any finite connected graph $\Gamma$
\begin{enumerate}[(a)]
\item $\Gamma$-comparison holds in any metric space,  or
\item $C_4\prec \Gamma$, or
\item $T_3\prec \Gamma$.
\end{enumerate}
\end{thm}

A connected graph will be called \emph{multipath} if it has an integer function $\ell$ on the set of the vertex set such that 
$v$ is adjacent to $w$ if and only if $$|\ell(v)-\ell(w)|\le 1.$$
The value $\ell(w)$ will be called \emph{level} of the vertex $w$.
A multipath is completely described by a sequence of integers that give the number of vertexes on each level.
\begin{figure}[ht!]
\centering
\includegraphics{mppics/pic-20}
\end{figure}
For example, the multipath shown on the diagram can be described by the sequence $(3,1,2,1,2,2,1,1,1,2)$. 

\begin{thm}{Lemma}
Let $\Gamma$ be a connected finite graph such that $C_4\nprec\Gamma$ and  $T_3\nprec\Gamma$.
Then $\Gamma$ is a multipath.
\end{thm}

\parit{Proof of the lemma.}
Let us denote by $|\ -\ |_\Gamma$ the path metric on the vertex set of $\Gamma$;
it is equal to the number of edges in a shortest path connecting two vertices.
Note that it is sufficient to show that 
\[|u-w|_\Gamma\ge |u-v|_\Gamma\ge|v-w|_\Gamma\ge 2
\quad\Longrightarrow\quad |u-w|_\Gamma=|u-v|_\Gamma+|v-w|_\Gamma
\leqno({*})\]
for any theree vertices $u$, $v$, and $w$ in $\Gamma$.

Suppose $({*})$ does not hold for $u$, $v$, and $w$.
Let us pass to a minimal connected induced subgraph $\Delta$ of $\Gamma$ that contains $u$, $v$, and $w$ and $({*})$ does not hold.
\begin{figure}[ht!]
\centering
\includegraphics{mppics/pic-25}
\end{figure}
Note that $\Delta$ is either a cycle
or it has three paths from a vertex, say $o$, to each of $u$, $v$, and $w$ such that each of these paths does not visit the remaining vertices in the triple $u$, $v$, $w$.
In the first case $C_4\prec\Delta$, and in the second $T_3\prec\Delta$.
By the observation above $\Delta\prec\Gamma$ --- the lemma is proved.
\qeds




\begin{thm}{Proposition}\label{lem:multipath}
Let $\Gamma$ be a multipath with sequence $(k_1,\dots, k_m)$.
Suppose $C_4\nprec\Gamma$ and  $T_3\nprec\Gamma$.
Then 
\begin{enumerate}[(a)]
 \item\label{lem:multipath:5} If $m\ge 5$, then $k_i=1$ for any $3\le i\le m-2$.
 \item\label{lem:multipath:4} If $m= 4$, then $k_2=1$ or $k_3=1$.
 \item\label{lem:multipath:3} If $m= 3$, then $k_1=1$, $k_2=1$, or $k_3=1$.
\end{enumerate}

\end{thm}

\parit{Proof of the proposition.} Assuming the contrary in each case we get

\parit{(\ref{lem:multipath:5}).}
If $m\ge 5$, then the multipath $(1,1,2,1,1)$ is induced subgraph of~$\Gamma$.

\parit{(\ref{lem:multipath:4}).}
If $m=4$, then the multipath $(1,2,2,1)$ is induced subgraph of~$\Gamma$.

\parit{(\ref{lem:multipath:3}).}
If $m=3$, then the multipath $(2,2,2)$ is induced subgraph of~$\Gamma$.

\begin{figure}[ht!]
\centering
\includegraphics{mppics/pic-30}
\end{figure}

In each case, we arrive at a contradiction by applying edge shrinking to the marked edges as shown on the diagram.
\qeds


It remains to show that $\Gamma$-comparison holds in any metric space for every multipath $\Gamma$ described in \ref{lem:multipath}.
This is done by describing coordinates of points in model configuration on the real line.

Each edge of $\Gamma$ comes with weight --- the distance between the endpoints in $X$.
Define the distance $\|v-w\|_\Gamma$ as the minimal total weight of path connecting $v$ to $w$ in $\Gamma$.
Note that 
\[\|v-w\|_\Gamma\ge |v-w|_X\]
for any $v$ and $w$.

If $m\le 2$ then $\Gamma$ is a complete graph.
In this case $\Gamma$-comparison is trivial.
It remains to consider cases $m\ge3$.

Let us choose a special vertex $v$ in $\Gamma$ that is unique on its level.
Namely, if $m\ge 5$, then $v$ is on the third level.
If $m=4$, then we can assume that $k_2=1$; in this case choose $v$ on the second level.
Finally, if $m=3$, let $v$ be any vertex that is unique on its level; it exists by the proposition. 

For every vertex $v_i$, let
\[\tilde v_i=\pm \|v-v_i\|_\Gamma,\]
where the sign is plus if $w$ has higher level and minus otherwise.
By triangle inequality, the obtained configuration $\tilde v_1,\dots,\tilde v_n$ meets the condition of $\Gamma$-comparison.
\qeds




\parbf{Acknowledgments.}
The first author was partially supported by the Russian Foundation for Basic Research grant 20-01-00070, the second author was partially supported by the National Science Foundation grant DMS-2005279.

{\sloppy
\printbibliography[heading=bibintoc]
\fussy
}

\Addresses
\end{document}

Mathematics Research Reports?
