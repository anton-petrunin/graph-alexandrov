\documentclass{article}
\usepackage{quad}

\def\thetitle{Graph comparison meets Alexandrov}
\def\theauthors{Nina Lebedeva and Anton Petrunin}

\hypersetup{pdftitle={\thetitle},
pdfauthor={\theauthors}}

\newcommand{\Addresses}{{\bigskip\footnotesize

\noindent Nina Lebedeva,
\par\nopagebreak
 \textsc{Saint Petersburg State University, 7/9 Universitetskaya nab., St. Petersburg, 199034, Russia}
\par
\nopagebreak
 \textsc{St. Petersburg Department of V.A. Steklov Institute of Mathematics of the Russian Academy of Sciences, 27 Fontanka nab., St. Petersburg, 191023, Russia}
  \par\nopagebreak
  \textit{Email}: \texttt{lebed@pdmi.ras.ru}

\medskip

\noindent   Anton Petrunin, 
\par\nopagebreak
 \textsc{Math. Dept. PSU, University Park, PA 16802, USA.}
  \par\nopagebreak
  \textit{Email}: \texttt{petrunin@math.psu.edu}
  
}}

\begin{document}
%\pagestyle{empty}\renewcommand\includegraphics[2][{}]{}


\title{\thetitle}
\author{\theauthors}

\date{}
\maketitle
\begin{abstract}
Graph comparison is a certain type of condition on metric space encoded by a finite graph.
We show that any nontrivial graph comparison implies one of two Alexandrov's comparisons.
The proof gives a complete description of graphs with trivial graph comparisons.
\end{abstract}

The notion of graph comparison was introduced in \cite{lebedeva-petrunin-zolotov}.
It was studied further in \cite{toyoda,toyoda2019,lebedeva-petrunin-CBB,lebedeva,lebedeva-petrunin,lebedeva-petrunin-octahedron}.
Let us mention some of the results.
\begin{itemize}
\item Graph comparisons for the tripod and four-cycle capture nonnegative and nonpositive curvature in the sense of Alexandrov; see below.
\item Graph comparison for certain trees is used to formulate a stronger version of the so-called \emph{Lang--Schroeder--Sturm inequality} \cite{lang-schroeder, sturm, lebedeva-petrunin-CBB}.
\item The all-tree comparison gives a metric description of target spaces of submetries from subsets of Hilbert space \cite{lebedeva-petrunin-zolotov}.
\item For a certain tree, graph comparison has tight relation with the so-called \emph{MTW condition} that was introduced by Xi-Nan Ma, Neil Trudinger, and Xu-Jia Wang~\cite{lebedeva-petrunin-zolotov,ma-trudinger-wang}.
\item Octahedron comparison holds in products of trees~\cite{lebedeva-petrunin-octahedron}.
\end{itemize}
We will show that any nontrivial graph comparison implies one of two Alexandrov's comparisons.

Let us start with the definition.
Suppose $\Gamma$ is a graph with vertices $v_1,\dots,v_n$.
We write $v_i\sim v_j$ (or $v_i\nsim v_j$) if $v_i$ is adjacent (respectively nonadjacent) to $v_j$.

A metric space $X$ meets the \emph{$\Gamma$-comparison} if for any $n$ points in $X$ labeled by vertices of $\Gamma$ there is a model configuration $\tilde v_1,\dots,\tilde v_n$ in the Hilbert space $\HH$ such that 
\begin{align*}
v_i\sim v_j\quad&\Longrightarrow\quad|\tilde v_i-\tilde v_j|_{\HH}\le | v_i-v_j|_{X},
\\
v_i\nsim v_j\quad&\Longrightarrow\quad|\tilde v_i-\tilde v_j|_{\HH}\ge | v_i-v_j|_{X};
\end{align*}
here $|\ -\ |_X$ denotes distance in the metric space~$X$.
(Note that $v_i$ may refer to a vertex in $\Gamma$ and to the corresponding point in $X$.)

\begin{wrapfigure}{r}{35 mm}
\vskip-6mm
\centering
\includegraphics{mppics/pic-10}
\end{wrapfigure}

Denote by $T_3$ and $C_4$ and tripod and four-cycle shown on the diagram.
The $C_4$-comparison is equivalent to nonnegative curvature,
and $T_3$-comparison is equivalent to the nonpositive curvature in the sense of Alexandrov \cite{lebedeva-petrunin-zolotov}.
These definitions are usually applied to length spaces, but they can be applied to general metric spaces;
the latter convention is used in \cite{alexander2019alexandrov}.

\begin{thm}{Theorem}
Let $\Gamma$ be an arbitrary finite graph.
Then either $\Gamma$-comparison holds in any metric space,
or it implies $C_4$- or $T_3$-comparison.
\end{thm}

The following corollary describes all graphs $\Gamma$ with trivial $\Gamma$-comparison;
it follows from the proof of the theorem.

\begin{thm}{Corollary} Let $\Gamma$ be a finite connected graph.
Suppose that $\Gamma$-comparison is trivial;
that is, it holds in any metric space.
Then $\Gamma$ can be constructed from a path $P_{\ell}$ of length $\ell\ge 0$ and two complete graphs $K_{m_1}$, $K_{m_2}$ by attaching $k_1$ vertices of $K_{m_1}$ to the left end of $P_{\ell}$ and $k_2$ vertices of $K_{m_2}$ to the right end of~$P_{\ell}$.
\end{thm}

Note that the graph $\Gamma$ in the corollary is described by five integers $(m_1$, $k_1$, $\ell$, $k_2$, $m_2)$ such that $\ell \geqslant 0$, $m_i\geqslant k_i\ge 0$, and $k_i>0$ if $m_i>0$ for each $i$.
Examples of such graphs and their 5-arrays are shown below.
\begin{figure}[ht!]
\centering
\includegraphics{mppics/pic-15}
\end{figure}

\bigskip

\parbf{Proof.}
Suppose $\Gamma$ has connected components $\Gamma_1,\dots,\Gamma_k$.
Observe that $\Gamma$-comparison holds in a metric space $X$ if and only if so does every $\Gamma_i$-comparison.
Therefore we can assume that $\Gamma$ is connected.

Suppose $\Gamma$ is a graph with vertices $v_1,\dots,v_n$ as before.
Let $e$ be an edge in $\Gamma$; we can assume that it connects $v_1$ to $v_2$.
Remove $v_1$ and $v_2$ from $\Gamma$ and add a new vertex $w$ such that for any other vertex $u$ we have
\begin{itemize}
 \item if $u\sim v_1$ and $u\sim v_2$, then $u\sim w$;
 \item if $u\nsim v_1$ and $u\nsim v_2$, then $u\nsim w$;
 \item in the remaining cases we can choose arbitrarily $u\sim w$ or $u\nsim w$.
\end{itemize}
Denote the obtained graph by $\Gamma'$.

Applying the definition of $\Gamma$-comparison assuming that $v_1=v_2$ in $X$, we get the following.

\begin{thm}{Claim}
If $\Gamma$-comparison holds in a metric space $X$, then so does $\Gamma'$-comparison.

\end{thm}

The operation that produces $\Gamma'$ from $\Gamma$ will be called \emph{edge shrinking}.
If a graph $\Delta$ can be obtained from $\Gamma$ applying edge shrinking several times, then we will write $\Delta\prec \Gamma$.

Note that the claim implies the following two statements:
\begin{itemize}
 \item If $\Delta$ is an induced subgraph of a connected finite graph $\Gamma$, then $\Delta\prec \Gamma$.
 \item If $\Delta\prec \Gamma$, then $\Gamma$-comparison implies $\Delta$-comparison.
\end{itemize}
Taking all the above into account, we get the following reformulation of the theorem.

\begin{thm}{Reformulation}
For any finite connected graph $\Gamma$
\begin{enumerate}[(a)]
\item $\Gamma$-comparison holds in any metric space,  or
\item $C_4\prec \Gamma$, or
\item $T_3\prec \Gamma$.
\end{enumerate}
\end{thm}

A connected graph will be called \emph{multipath} if it has an integer function $\ell$ on the set of the vertex set such that 
$v$ is adjacent to $w$ if and only if $$|\ell(v)-\ell(w)|\le 1.$$
The value $\ell(w)$ will be called the \emph{level} of the vertex $w$.
Multipath is completely described by a sequence of integers that give the number of vertexes on each level.
\begin{figure}[ht!]
\centering
\medskip
\includegraphics{mppics/pic-20}
\medskip
\end{figure}
An example of a multipath with its sequence is shown on the diagram. 

\begin{thm}{Lemma}
Let $\Gamma$ be a connected finite graph such that $C_4\nprec\Gamma$ and  $T_3\nprec\Gamma$.
Then $\Gamma$ is a multipath.
\end{thm}

\parit{Proof of the lemma.}
Let us denote by $|\ -\ |_\Gamma$ the path metric on the vertex set of $\Gamma$;
it is equal to the number of edges in a shortest path connecting two vertices.
Note that it is sufficient to show that 
\[|u-w|_\Gamma\ge |u-v|_\Gamma\ge|v-w|_\Gamma\ge 2
\quad\Longrightarrow\quad |u-w|_\Gamma=|u-v|_\Gamma+|v-w|_\Gamma
\leqno({*})\]
for any three vertices $u$, $v$, and $w$ in $\Gamma$.

Suppose $({*})$ does not hold for $u$, $v$, and $w$.
Let us pass to a minimal connected induced subgraph $\Delta\ni u, v, w$ of $\Gamma$ such that $({*})$ still does not hold in $\Delta$.
\begin{figure}[ht!]
\centering
\includegraphics{mppics/pic-25}
\end{figure}
Note that $\Delta$ is either a cycle
or it has three paths from a vertex, say $o$, to each of $u$, $v$, and $w$ such that each of these paths do not visit the remaining vertices in the triple $u$, $v$, $w$.
In these cases, we have $C_4\prec\Delta$ and $T_3\prec\Delta$ respectively.
By the observation above $\Delta\prec\Gamma$ --- the lemma is proved.
\qeds




\begin{thm}{Proposition}\label{lem:multipath}
Let $\Gamma$ be a multipath with sequence $(k_0,\dots, k_m)$.
Suppose $C_4\nprec\Gamma$ and  $T_3\nprec\Gamma$.
Then 
\begin{enumerate}[(a)]
 \item\label{lem:multipath:5} If $m\ge 4$, then $k_2=\dots=k_{m-2}=1$.
 \item\label{lem:multipath:4} If $m= 3$, then $k_1=1$ or $k_2=1$.
 \item\label{lem:multipath:3} If $m= 2$, then $k_0=1$, $k_1=1$, or $k_2=1$.
\end{enumerate}

\end{thm}

\parit{Proof of the proposition.} Assuming the contrary in each case we get

\parit{(\ref{lem:multipath:5}).}
If $m\ge 4$, then multipath $(1,1,2,1,1)$ is an induced subgraph of~$\Gamma$.

\parit{(\ref{lem:multipath:4}).}
If $m=3$, then multipath $(1,2,2,1)$ is an induced subgraph of~$\Gamma$.

\parit{(\ref{lem:multipath:3}).}
If $m=2$, then multipath $(2,2,2)$ is an induced subgraph of~$\Gamma$.

In each case, we arrive at a contradiction by applying edge shrinking to the marked edges as shown on the diagram.

\begin{figure}[h!]
\centering
\includegraphics{mppics/pic-30}
\end{figure}
\qedsf


It remains to show that $\Gamma$-comparison holds in any metric space for every multipath $\Gamma$ described in \ref{lem:multipath}.
This is done by prescribing the coordinates for the needed model configuration on the real line.

Each edge of $\Gamma$ comes with weight --- the distance between the endpoints in $X$.
Define the distance $\|v-w\|_\Gamma$ as the minimal total weight of paths connecting $v$ to $w$ in $\Gamma$.
Note that 
\[\|v-w\|_\Gamma\ge |v-w|_X\]
for any $v$ and $w$.

If $m\le 1$ then $\Gamma$ is a complete graph.
In this case, $\Gamma$-comparison is trivial.
It remains to consider cases $m\ge2$.

Let us choose a special vertex $w$ that is unique on its level and not too far from the middle of $\Gamma$.
Namely, if $m\ge 4$, then choose $w$ on the second level; by the proposition it is unique on its level.
If $m=3$, then by the proposition we can assume that $k_2=1$; in this case choose $w$ on the second level.
Finally, if $m=2$, let $w$ be any vertex that is unique on its level; it exists by the proposition. 

For every vertex $v_i$, let
\[\tilde v_i=\pm \|w-v_i\|_\Gamma,\]
where the sign is plus if $v_i$ has a higher level than $w$ and minus otherwise.
By the triangle inequality, the obtained configuration $\tilde v_1,\dots,\tilde v_n$ meets the condition of $\Gamma$-comparison.
\qeds


\parbf{Acknowledgments.}
The first author was partially supported by the Russian Foundation for Basic Research grant 20-01-00070; the second author was partially supported by the National Science Foundation grant DMS-2005279.

{\sloppy
\printbibliography[heading=bibintoc]
\fussy
}

\Addresses
\end{document}

Mathematics Research Reports?
